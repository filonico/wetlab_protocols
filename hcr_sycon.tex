\documentclass[10pt]{report}
\usepackage[utf8]{inputenc}

% DOCUMENT STYLE
\usepackage[a4paper,
  left=1.5cm, right=1.5cm, top=2.7cm, bottom=2cm,
  headsep=\dimexpr2.5cm-55pt\relax,
  headheight=55pt]{geometry}
\linespread{1.5}                                    % regulate line spacing
\usepackage{setspace}
\usepackage[parfill]{parskip}                       % remove indentation of first line
\usepackage[T1]{fontenc}
\usepackage{xcolor}
\usepackage{enumitem}
\usepackage{float}

% FONTS
\usepackage{fontspec}
\setmainfont{RobotoSerif}[
  Path = ./font/ ,
  Extension = .ttf ,
  UprightFont = *-Regular ,
  ItalicFont = *-Italic ,
  BoldFont = *-Bold ,
  BoldItalicFont = *-BoldItalic
]

% package to generate random textcomp
\usepackage{blindtext}

% FORMAT HEADERS AND FOOTERS
\usepackage{fancyhdr}
\pagestyle{fancy} % page style to fancy to print header
\rhead{\footnotesize \emph{Last update}: \today}
\lhead{\footnotesize \em \nouppercase{\leftmark{} \textendash{} \rightmark}}

% CHEMICAL FORMULAE AND SPECIAL SYMBOLS
\usepackage[version=4]{mhchem}
\usepackage{textcomp}
\usepackage{siunitx}
\sisetup{
  detect-all = true,                                    % detect and follow surrounding font changes
  range-phrase= \textendash{}, range-units = single,               % set range format
  per-mode = symbol,                                    % set "/" as format for fraction units
  uncertainty-mode = separate,                          % seprate uncertainty (+-) from number
  separate-uncertainty-units = single,                  % put units only after uncertainty
  group-digits = integer,                               % apply digit grouping just for the integer part 
  group-separator = {,},                                % separator between digits
  group-minimum-digits = 4,                             % set the minimun number of digits to apply goruping to 4
  bracket-unit-denominator = false,                     % do not put brackets in denominator
  list-units = single,                                  % in list mode, put units only at the end
  exponent-mode = threshold,                            % automatically typeset with scientific notation (default threshold of -3:3)
  print-unity-mantissa = false                          % do not print mantissa if it is 1
  %round-mode = places,
  %round-precision = 3
  }
\DeclareSIUnit{\nothing}{\relax}                  % define unit to typset just prefixes
\DeclareSIUnit{\molar}{M}                         % define unit for molar cocnentration (M)

\newcommand{\per}{$\times$}
\newcommand{\NA}{\textendash{}}
\newcommand{\doublecurlyquotes}[1]{“#1”}
\newcommand{\singlecurlyquotes}[1]{‘#1’}
\newcommand{\curlyapostrophe}{’}

% percentages
\newcommand{\hundredpercent}{\qty{100}{\percent}}
\newcommand{\fivepercent}{\qty{5}{\percent}}
\newcommand{\onepercent}{\qty{1}{\percent}}
\newcommand{\zeroonepercent}{\qty{0.1}{\percent}}

% volumes
\newcommand{\twohunmicrol}{\qty{200}{\ul}}
\newcommand{\fivehunmicrol}{\qty{500}{\ul}}
\newcommand{\onemil}{\qty{1}{\ml}}

% temperatures
\newcommand{\fourdegree}{\qty{4}{\degreeCelsius}}
\newcommand{\thirtysevendegree}{\qty{37}{\degreeCelsius}}
\newcommand{\minustwenty}{\qty{-20}{\degreeCelsius}}

% time
\newcommand{\tenmin}{\qty{10}{\minute}}
\newcommand{\quarter}{\qty{15}{\minute}}
\newcommand{\halfhour}{\qty{30}{\minute}}

% reagents
\newcommand{\pbs}{1\per{} PBS}
\newcommand{\hs}{0.25\per{} HS}
\newcommand{\ssct}{5\per{} SSCT}

% others
\newcommand{\qtytimes}[3]{#1 \per{} \qty{#2}{#3}}
\newcommand{\noexponentnum}[1]{\num[exponent-mode = fixed, fixed-exponent = 0]{#1}}

% DATES
\usepackage[datesep=/, calc, en-GB]{datetime2}
\usepackage{fmtcount}

% define the style for CV, that is: Jan 1st, 2024
\DTMnewdatestyle{cvdateformat}{%
  \renewcommand{\DTMdisplaydate}[4]{%
  \DTMshortmonthname{##2} \protect\ordinalnum{##3}, \number##1 }%
  \renewcommand{\DTMDisplaydate}{\DTMdisplaydate}%
}

% GLOSSARIES, ACRONYMS, AND ABBREVIATIONS
\usepackage[acronym, nonumberlist, toc, nomain, nopostdot, shortcuts]{glossaries-extra}
%\loadglsentries{sections/abbreviations}	
% define command to print brackets when first usage is within parenthesis
\newcommand*{\pac}[2][]{\ifglsused{#2}{\acs[#1]{#2}}{%
 \glsunset{#2}%
 \acl[#1]{#2} [\acs[#1]{#2}]}}

% DEFINE TITLE STYLES
\makeatletter 
\renewcommand\maketitle{
{\begin{center}
{\Large \bfseries \@title }\\
{\small \emph{First created}: \DTMdisplaydate{2024}{12}{5}{-1}\hfill%
\setstretch{1.0} \small \hfill \emph{Last update}: \today}
%{\Large  \@author}\\[4ex] 
\end{center}}} % Note the extra }
\makeatother

\usepackage{titlesec}
%\titleformat{⟨command⟩}[⟨shape⟩]{⟨format⟩}{⟨label⟩}{⟨sep⟩}{⟨before-code⟩}[⟨after-code⟩]
\titleformat{\section}{\large\bfseries\uppercase}{\thesection}{}{\centering}
\titleformat{\subsection}{\large\bfseries}{\thesubsection}{}{}

% DEFINE BUTTON-STYLE BOXES
%\usepackage{tcolorbox}
%\tcbset{on line}

%\newcommand{\insertboxes}[3]{{\bfseries \tcbox{#1} \tcbox{#2} \tcbox{#3}}}

% DEFINE LEFT EMPHASIS-RULE
\usepackage{mdframed}
\newmdenv[
  topline=false,
  bottomline=false,
  rightline=false,
  linecolor=gray
]{leftrule}
\newcommand{\alert}[1]{{\itshape \small \setstretch{1.0} \begin{leftrule} \textcolor{gray}{#1} \end{leftrule}}}

% STEP SYNTAX
% wash steps: #1=action; #2=time; #3=quantity; #4=reagent; #5=temperature
\newcommand{\step}[5]{\MakeUppercase #1 samples for \underline{#2} with #3 of \textbf{#4} at #5.}
\newcommand{\note}[1]{{\itshape \small \textcolor{gray}{Note: #1}}}


% DEFINE TITLE STYLES
\makeatletter 
\renewcommand\maketitle{
{\begin{center}
{\Large \bfseries \@title }\\
{\small \emph{First created}: \DTMdisplaydate{2024}{12}{5}{-1}\hfill%
\setstretch{1.0} \small \hfill \emph{Last update}: \today}
%{\Large  \@author}\\[4ex] 
\end{center}}} % Note the extra }
\makeatother

%-------------------------------------------------------------------------

\begin{document}

%\sisetup{parse-numbers = false}
\DTMsetdatestyle{cvdateformat}

\title{\vspace{-3em} PROVA Multiplexed mRNA \textit{in-situ} Hybridization Chain Reaction (HCR) in \emph{Sycon ciliatum}}
\maketitle
\thispagestyle{plain}
\markboth{HCR in \emph{Sycon}}{}

\section*{\vspace{-1em} \textendash{} Protocol \textendash{}}
\markright{Protocol}

\subsection*{Day 1 \textendash{} Sample preparation (version with xylene clearing)}

\alert{This protocol refers to samples that have been already fixed and stored in pure methanol at \minustwenty.}

\begin{enumerate}[series = steps]
	\item \step{wash}{\quarter}{\fivehunmicrol{} (or \onemil)}{\textbf{\hundredpercent{} ethanol}}{RT}
	\item \step{clear}{\halfhour}{\fivehunmicrol{} (or \onemil)}{\textbf{\qty{50}{\percent} xylene in ethanol}}{RT}
\end{enumerate}

\bigskip\alert{Always use xylene in glass containers (as it dissolves plastic) and under a chemical safety hood.}

\begin{enumerate}[resume = steps]
	\item \step{wash}{\quarter}{\fivehunmicrol{} (or \onemil)}{ice-cold \hundredpercent{} ethanol}{\fourdegree}
	\item Wash samples for \underline{\quarter} each with a graded series of \fivehunmicrol{} (or \onemil) \textbf{ice-cold ethanol (\qtylist{75;50;25}{\percent}) in \hs{}} at \fourdegree.
	\item \step{wash}{\tenmin}{\fivehunmicrol{} (or \onemil)}{ice-cold \hs}{\fourdegree}
	\item \step{wash}{\tenmin}{\fivehunmicrol{} (or \onemil)}{ice-cold \pbs}{\fourdegree}
	\item \step{post-fix}{\halfhour}{\fivehunmicrol{} (or \onemil)}{ice-cold fixative solution}{\fourdegree}
\end{enumerate}

\bigskip\alert{Pre-heat probe-hybridization buffer to \thirtysevendegree{} (you will need \qty{700}{\ul} per tube [\fivehunmicrol for the pre-hybridization, and \qty{200}{\ul} for the probe solution]).}

\begin{enumerate}[resume = steps]
	\item \step{wash}{\qtytimes{4}{10}{\minute}}{\fivehunmicrol{} (or \onemil)}{PBST}{\fourdegree}
\end{enumerate}

\subsection*{Day 1 \textendash{} Sample preparation (version with proteinase-K permeabilisation)}

\alert{This protocol refers to samples that have been already fixed and stored in pure methanol at \minustwenty.}

\begin{enumerate}[series = steps]
	\item \step{wash}{\quarter}{\fivehunmicrol{} (or \onemil)}{ice-cold \hundredpercent{} ethanol}{\fourdegree}
	\item Wash samples for \underline{\quarter} each with a graded series of \fivehunmicrol{} (or \onemil) \textbf{ice-cold ethanol (\qtylist{75;50;25}{\percent}) in PBST} at \fourdegree.
	\item \step{rock}{\tenmin}{\fivehunmicrol{} (or \onemil)}{\qty{7.5}{\ug\per\ml} proteinase K in PBST}{\thirtysevendegree}
	\item \step{wash}{\tenmin}{\fivehunmicrol{} (or \onemil)}{PBST}{RT}
	\item \step{quench}{\tenmin}{\fivehunmicrol{} (or \onemil)}{\qty{2}{\mg\per\ml} glycine in PBST}{RT}
	\item \step{wash}{\qtytimes{3}{10}{\minute}}{\fivehunmicrol{} (or \onemil)}{PBST}{RT}
	\item Acetylate samples for \underline{\quarter} each with a graded series of \fivehunmicrol{} (or \onemil) \textbf{\qty{0.1}{\molar} triethanolamine containing \qtylist{0;1.5;3}{\ul\per\ml} acetic anhydride} ar RT.
	\item \step{wash}{\qtytimes{2}{10}{\minute}}{\fivehunmicrol{} (or \onemil)}{PBST}{RT}
	\item \step{post-fix}{\halfhour}{\fivehunmicrol{} (or \onemil)}{fixative solution}{RT}
\end{enumerate}

\bigskip\alert{Pre-heat probe-hybridization buffer to \thirtysevendegree{} (you will need \qty{700}{\ul} per tube [\fivehunmicrol for the pre-hybridization, and \twohunmicrol{} for the probe solution]).}

\begin{enumerate}[resume = steps]
	\item \step{wash}{\qtytimes{4}{10}{\minute}}{\fivehunmicrol{} (or \onemil)}{PBST}{RT}
\end{enumerate}


\subsection*{Day 1 \textendash{} HCR detection stage}
\begin{enumerate}[series = steps]
	\item \step{pre-hybridize}{\halfhour}{\fivehunmicrol}{pre-warmed probe hybridization buffer}{\thirtysevendegree}
\end{enumerate}

\bigskip\alert{If reusing probes, pre-warm probe solution to \thirtysevendegree, then skip to Step 3.}

\alert{At this step, samples can be stored in probe hybridization buffer at \minustwenty{} for several months.}

\begin{enumerate}[resume = steps]
	\item Prepare probe solution by adding \qty{0.8}{\ul} (\qty{0.8}{\pmol}) of each \textbf{probe set \qty{1}{\micro\molar} stock solution} to \twohunmicrol{} of \textbf{probe hybridization buffer} (final concentration of \qty{4}{\nano\molar}) at \thirtysevendegree.\\
	      \note{if using \fivehunmicrol{} of probe solution, add \qty{2}{\ul} (\qty{2}{\pmol}) of each probe set.}
	\item \step{incubate}{\qty{> 12}{\hour} (overnight)}{\twohunmicrol}{pre-warmed probe solution}{\thirtysevendegree}
\end{enumerate}

\bigskip\alert{Pre-heat probe wash buffer at \thirtysevendegree{} (you will need \fivehunmicrol per tube per 4 washes).}


\subsection*{Day 2 \textendash{} HCR amplification stage}

\alert{Pre-equilibrate amplification buffer to room temperature (you will need \qty{700}{\ul} per tube [\fivehunmicrol{} for pre-amplification, and \twohunmicrol{} for the hairpin solution]).}
\begin{enumerate}[series = steps]
	\item \step{wash}{\qtytimes{4}{20}{\minute}}{\fivehunmicrol}{probe wash buffer}{\thirtysevendegree}
\end{enumerate}

\bigskip\alert{Probe solution can be saved and reused for 2--3 times; store at \minustwenty.}

\begin{enumerate}[resume = steps]
	\item \step{wash}{\qtytimes{3}{5}{\minute}}{\fivehunmicrol{} (or \onemil)}{\ssct}{RT}
	\item \step{pre-amplify}{\halfhour}{\fivehunmicrol}{amplification buffer}{RT}
	\item Snap cool in separate tubes \qty{4}{\ul} of \textbf{hairpin H1} (\qty{30}{\pmol}) and \qty{4}{\ul} of \textbf{hairpin H2} (\qty{30}{\pmol}) \textbf{\qty{3}{\micro\molar} stock solutions}: heat at \qty{95}{\degreeCelsius} for \qty{90}{\s} and cool to RT in the dark for \halfhour.
	      %\note{doubling hairpin concentration can help to boost signal.}
\end{enumerate}

\bigskip\alert{If reusing hairpin solutions, heat at \qty{95}{\degreeCelsius} and cool at RT, then skip at Step 6.}

\begin{enumerate}[resume = steps]
	\item Prepare hairpin solution by adding \textbf{snap-cooled H1 hairpins} and \textbf{snap-cooled H2 hairpins} to \twohunmicrol{} of \textbf{amplification buffer} (final concentration of each hairpin of \qty{60}{\nano\molar}) at RT.\\
	      \note{if using \qty{100}{\ul} of amplification buffer, add \qty{2}{\ul} (\qty{15}{\pmol}) of each hairpin.}\\
	      \note{if using \qty{500}{\ul} of amplification buffer, add \qty{10}{\ul} (\qty{75}{\pmol}) of each hairpin.}
	\item \step{incubate}{\qty{>12}{\hour} (overnight)}{\twohunmicrol}{hairpin solution}{RT}
\end{enumerate}

\subsection*{Day 3 \textendash{} HCR conclusion}

\begin{enumerate}[series = steps]
	\item \step{wash}{\qtytimes{2}{5}{\minute}, \qtytimes{2}{30}{\minute}, and \qtytimes{1}{5}{\minute}}{\fivehunmicrol{} (or \onemil)}{\ssct}{RT}
\end{enumerate}

\bigskip\alert{Hairpin solution can be saved and reused for 2--3 times; store at \minustwenty.}
\alert{At this step, samples can be stored in the dark in \ssct for several days at room temperature.}

\subsection*{Day 3 \textendash{} Sample mounting}

\begin{enumerate}[series = steps]
	\item \step{wash}{\qtytimes{3}{5}{\minute}}{\fivehunmicrol{} (or \onemil)}{\pbs}{RT}
\end{enumerate}

\bigskip\alert{If using VECTASHIELD\textregistered{} PLUS Antifade Mounting Medium with DAPI (H-2000), skip directly to Step 4.}

\begin{enumerate}[resume = steps]
	\item \step{incubate}{\qty{20}{\minute}}{\fivehunmicrol{} (or \onemil)}{DAPI staining solution}{RT}
	\item \step{wash}{\qtytimes{3}{5}{\minute}}{\fivehunmicrol{} (or \onemil)}{\pbs}{RT}
	\item \step{incubate}{\qtyrange{30}{60}{\minute}}{\onemil}{\qty{50}{\percent} glycerol in \pbs}{RT}
	\item \step{incubate}{\qtyrange{30}{60}{\minute}}{\onemil}{\qty{75}{\percent} glycerol in \pbs}{RT}
\end{enumerate}

\bigskip\alert{Maintaining samples at pH 7.40 is critical for the stability of fluorophores and long-term storage of mRNA \textit{in-situ} HCR samples.}
\alert{At this step, samples can be stored in the dark in \qty{75}{\percent} glycerol for several month at \fourdegree.}

\begin{enumerate}[resume = steps]
	\item Collect precipitated samples from the \qty{75}{\percent} glycerol solution and place them on a cleaned (bridged, if necessary) slide.
	\item Add \qtyrange{\sim 15}{20}{\ul} of mounting medium (or VECTASHIELD\textregistered{} PLUS Antifade Mounting Medium with DAPI [H-2000]) directly on samples.
\end{enumerate}

\bigskip\alert{Adjust the amount of embryo suspension and mounting medium according to the cover glass dimensions, the quantity of samples, and the thickness of the bridge.}

\begin{enumerate}[resume = steps]
	\item	Seal the slide with nail polish and store in the dark at \fourdegree.
\end{enumerate}

\clearpage

\section*{\textendash{} Recipes \textendash{}}
\setstretch{0.95}
\markright{Recipes}

\subsection*{Fixative solution}

\begin{table}[H]
	%\setstretch{0.95}
	\centering
	\begin{tabular}{r
		S[table-format = 4.1, exponent-mode = fixed, fixed-exponent = 0, round-mode = places, round-precision = 1]@{\,} % disable scientific notation
		l
		S[table-format = 3.2, table-align-text-post = false, exponent-mode = fixed, fixed-exponent = 0, round-mode = places, round-precision = 2] % disable scientific notation
		c
		}
		\textbf{Compound}                        & \multicolumn{2}{c}{\textbf{Quantity}} & \multicolumn{1}{c}{\textbf{\begin{tabular}[c]{@{}c@{}}Molar mass\\ (\unit{\g\per\mole})\end{tabular}}} & \multicolumn{1}{l}{\textbf{\begin{tabular}[c]{@{}c@{}}Final concentration\\ (for \qty{100}{\ml})\end{tabular}}}                        \\*[0.4cm]
		\qtyrange{36}{38}{\percent} formaldehyde & 3.7                                   & \unit{\ml}                                                                                             & 30.031                                                                                                          & \qty{3.7}{\percent}  \\
		\qty{25}{\percent} gluteraldehyde        & 0.2                                   & \unit{\ml}                                                                                             & 100.12                                                                                                          & \qty{0.05}{\percent} \\
		10\per{} PBS                             & 10                                    & \unit{\ml}                                                                                             & \NA                                                                                                             & 1\per{}              \\
		Ultrapure water                          & \multicolumn{2}{c}{\NA}               & \NA                                                                                                    & \NA
	\end{tabular}
\end{table}

\begin{enumerate}
	\item Add \qty{3.7}{\ml} of \qtyrange{36}{38}{\percent} formaldehyde to a graduated cylinder.
	\item Add \qty{0.2}{\ml} of \qty{25}{\percent} gluteraldehyde to a graduated cylinder.
	\item Fill up to \qty{100}{\ml} with ultrapure water.
	\item Store the fixative solution at \fourdegree.
\end{enumerate}

\subsection*{1\per{} Holtfreter\curlyapostrophe s solution (HS)}

\begin{table}[H]
	\centering
	\begin{tabular}{r
		S[table-format = 4.2, exponent-mode = fixed, fixed-exponent = 0, round-mode = places, round-precision = 2]@{\,} % disable scientific notation
		l
		S[table-format = 3.2, table-align-text-post = false, exponent-mode = fixed, fixed-exponent = 0, round-mode = places, round-precision = 2] % disable scientific notation
		S[table-format = 3.5, table-align-text-post = false, exponent-mode = fixed, fixed-exponent = 0, round-mode = places, round-precision = 5] % disable scientific notation
		}
		\textbf{Compound}           & \multicolumn{2}{c}{\textbf{Quantity}} & \multicolumn{1}{c}{\textbf{\begin{tabular}[c]{@{}c@{}}Molar mass\\ (\unit{\g\per\mole})\end{tabular}}} & \multicolumn{1}{l}{\textbf{\begin{tabular}[c]{@{}c@{}}Final concentration\\ (for \qty{1}{\l})\end{tabular}}}                          \\*[0.4cm]
		\ce{NaCl}                   & 3.46                                  & \unit{\g}                                                                                              & 58.44                                                                                                        & \qty{0.059}{\molar}    \\
		\ce{KCl}                    & 0.05                                  & \unit{\g}                                                                                              & 74.55                                                                                                        & \qty{0.00067}{\molar}  \\*[0.1cm]
		\ce{CaCl2}                  & 0.1                                   & \unit{\g}                                                                                              & 110.98                                                                                                       & \qty{0.00076}{\molar}  \\
		\textit{or} \ce{CaCl2*2H2O} & 0.1324                                & \unit{\g}                                                                                              & 120.037                                                                                                      & \qty{0.000901}{\molar} \\*[0.1cm]
		\ce{NaHCO3}                 & 0.2                                   & \unit{\g}                                                                                              & 84                                                                                                           & \qty{0.0024}{\molar}   \\
		Ultrapure water             & \multicolumn{2}{c}{\NA}               & \NA                                                                                                    & \NA
	\end{tabular}
\end{table}

\begin{enumerate}
	\item Dissolve solutes in \qty{800}{\ml} of ultrapure water, by continuous stirring.
	\item Fill up to \qty{1}{\l} with ultrapure water.
	\item Store the \hs indefinitely at RT.
\end{enumerate}

\subsection*{\pbs{} with \qty{0.1}{\percent} Tween 20 (PBST or PTw)}

\begin{table}[H]
	\centering
	\begin{tabular}{r
		S[table-format = 4.1, exponent-mode = fixed, fixed-exponent = 0, round-mode = places, round-precision = 1]@{\,} % disable scientific notation
		l
		S[table-format = 3.2, table-align-text-post = false, exponent-mode = fixed, fixed-exponent = 0, round-mode = places, round-precision = 2] % disable scientific notation
		c
		}
		\textbf{Compound} & \multicolumn{2}{c}{\textbf{Quantity}} & \multicolumn{1}{c}{\textbf{\begin{tabular}[c]{@{}c@{}}Molar mass\\ (\unit{\g\per\mole})\end{tabular}}} & \multicolumn{1}{l}{\textbf{\begin{tabular}[c]{@{}c@{}}Final concentration\\ (for \qty{50}{\ml})\end{tabular}}}                        \\*[0.4cm]
		Tween 20          & 50                                    & \unit{\ul}                                                                                             & 1227.54                                                                                                        & \qty{0.01}{\percent} \\
		20\per{} PBS      & 5                                     & \unit{\ml}                                                                                             & \NA                                                                                                            & 1\per{}              \\
		Ultrapure water   & \multicolumn{2}{c}{\NA}               & \NA                                                                                                    & \NA
	\end{tabular}
\end{table}

\begin{enumerate}
	\item Add \qty{5}{\ml} of 20\per{} PBS to a graduated cylinder.
	\item Add \qty{50}{\ul} of Tween 20.
	\item Fill up to \qty{50}{\ml} with ultrapure water.
\end{enumerate}

\subsection*{5\per{} SSC \qty{0.1}{\percent} Tween 20 (SSCT)}

\begin{table}[H]
	\centering
	\begin{tabular}{r
		S[table-format = 4.1, exponent-mode = fixed, fixed-exponent = 0, round-mode = places, round-precision = 1]@{\,} % disable scientific notation
		l
		S[table-format = 3.2, table-align-text-post = false, exponent-mode = fixed, fixed-exponent = 0, round-mode = places, round-precision = 2] % disable scientific notation
		c
		}
		\textbf{Compound} & \multicolumn{2}{c}{\textbf{Quantity}} & \multicolumn{1}{c}{\textbf{\begin{tabular}[c]{@{}c@{}}Molar mass\\ (\unit{\g\per\mole})\end{tabular}}} & \multicolumn{1}{l}{\textbf{\begin{tabular}[c]{@{}c@{}}Final concentration\\ (for \qty{50}{\ml})\end{tabular}}}                        \\*[0.4cm]
		Tween 20          & 50                                    & \unit{\ul}                                                                                             & 1227.54                                                                                                        & \qty{0.01}{\percent} \\
		20\per{} SSC      & 12.5                                  & \unit{\ml}                                                                                             & \NA                                                                                                            & 5\per{}              \\
		Ultrapure water   & \multicolumn{2}{c}{\NA}               & \NA                                                                                                    & \NA
	\end{tabular}
\end{table}

\begin{enumerate}
	\item Add \qty{12.5}{\ml} of 20\per{} SSC to a graduated cylinder.
	\item Add \qty{50}{\ul} of Tween 20.
	\item Fill up to \qty{50}{\ml} with ultrapure water.
\end{enumerate}

\subsection*{Proteinase K working solution}

\begin{table}[H]
	\centering
	\begin{tabular}{r
		S[table-format = 4.2, exponent-mode = fixed, fixed-exponent = 0, round-mode = places, round-precision = 2]@{\,} % disable scientific notation
		l
		S[table-format = 3.2, table-align-text-post = false, exponent-mode = fixed, fixed-exponent = 0, round-mode = places, round-precision = 2] % disable scientific notation
		c
		}
		\textbf{Compound}                                                                                              & \multicolumn{2}{c}{\textbf{Quantity}} & \multicolumn{1}{c}{\textbf{\begin{tabular}[c]{@{}c@{}}Molar mass\\ (\unit{\g\per\mole})\end{tabular}}} & \multicolumn{1}{l}{\textbf{\begin{tabular}[c]{@{}c@{}}Final concentration\\ (for \qty{10}{\ml})\end{tabular}}}                         \\*[0.4cm]
		\multicolumn{1}{c}{\begin{tabular}[c]{@{}r@{}}\qty{20}{\mg\per\ml} proteinase K\\ stock solution\end{tabular}} & 3.75                                  & \unit{\ul}                                                                                             & \NA                                                                                                            & \qty{7.5}{\ug\per\ml} \\*[0.4cm]
		PBST                                                                                                           & \multicolumn{2}{c}{\NA}               & \NA                                                                                                    & \NA
	\end{tabular}
\end{table}

\begin{enumerate}
	\item Add \qty{3.75}{\ul} of \qty{20}{\mg\per\ml} proteinase K stock solution to \qty{8}{\ml} of PBST.
	\item Fill up to \qty{10}{\ml} with PBST.
\end{enumerate}

\subsection*{1:500 DAPI staining solution}

\begin{table}[H]
	\centering
	\begin{tabular}{r
		S[table-format = 4.1, exponent-mode = fixed, fixed-exponent = 0, round-mode = places, round-precision = 1]@{\,} % disable scientific notation
		l
		S[table-format = 3.2, table-align-text-post = false, exponent-mode = fixed, fixed-exponent = 0, round-mode = places, round-precision = 2] % disable scientific notation
		c
		}
		\textbf{Compound} & \multicolumn{2}{c}{\textbf{Quantity}} & \multicolumn{1}{c}{\textbf{\begin{tabular}[c]{@{}c@{}}Molar mass\\ (\unit{\g\per\mole})\end{tabular}}} & \multicolumn{1}{l}{\textbf{\begin{tabular}[c]{@{}c@{}}Final concentration\\ (for \qty{500}{\ul})\end{tabular}}}         \\*[0.4cm]
		DAPI              & 1                                     & \unit{\ul}                                                                                             & \NA                                                                                                             & 1:500 \\
		\pbs{}            & 499                                   & \unit{\ul}                                                                                             & \NA                                                                                                             & \NA
	\end{tabular}
\end{table}

\clearpage

\section*{\textendash{} Resources \textendash{}}
\markright{Resources}

\begin{enumerate}
	\item \textbf{mRNA fluorescent \textit{in-situ} HCR in \textit{Sycon} spp.}
		\begin{itemize}
			\item \textbf{*}Fortunato, S., Adamski, M., Bergum, B., Guder, C., Jordal, S., Leininger, S., ... \& Adamska, M. (2012). Genome-wide analysis of the sox family in the calcareous sponge \textit{Sycon ciliatum}: multiple genes with unique expression patterns. \textit{EvoDevo}, \textit{3}, 1-11. \ulhref{https://doi.org/10.1186/2041-9139-3-14}{10.1186/2041-9139-3-14}
			\item Voigt, O., Adamski, M., Sluzek, K., \& Adamska, M. (2014). Calcareous sponge genomes reveal complex evolution of $\alpha$-carbonic anhydrases and two key biomineralization enzymes. \textit{BMC Evolutionary Biology}, \textit{14}, 1-19. \\ \ulhref{https://doi.org/10.1186/s12862-014-0230-z}{10.1186/s12862-014-0230-z}
			\item Voigt, O., Adamska, M., Adamski, M., Kittelmann, A., Wencker, L., \& W{\"o}rheide, G. (2017). Spicule formation in calcareous sponges: coordinated expression of biomineralization genes and spicule-type specific genes. \textit{Scientific Reports}, \textit{7}(1), 45658. \ulhref{https://doi.org/10.1038/srep45658}{10.1038/srep45658}
		\end{itemize}
	\item \textbf{mRNA \textit{in-situ} HCR in other sponges.}
		\begin{itemize}
			\item Kojima, C., \& Funayama, N. (2022). \textit{In situ} hybridization to identify stem cells in the freshwater sponge \textit{Ephydatia fluviatilis}. In: Blanchoud, S., \& Galliot, B. (eds.), \textit{Whole-Body Regeneration}, 335-346. \\ \ulhref{https://doi.org/10.1007/978-1-0716-2172-1}{10.1007/978-1-0716-2172-1}
			\item Larroux, C., Fahey, B., Adamska, M., Richards, G. S., Gauthier, M., Green, K., \& Degnan, B. M. (2008). Whole-mount \textit{in situ} hybridization in \textit{Amphimedon}. \textit{CSH protocols}, pdb-prot5096. \ulhref{https://cshprotocols.cshlp.org/content/2008/12/pdb.prot5096}{10.1101/pdb.prot5096}
			\item Musser, J. M., Schippers, K. J., Nickel, M., Mizzon, G., Kohn, A. B., Pape, C., \& Arendt, D. (2021). Profiling cellular diversity in sponges informs animal cell type and nervous system evolution. \textit{Science}, \textit{374}(6568), 717-723. \ulhref{https://doi.org/10.1126/science.abj2949}{10.1126/science.abj2949}
			\item Funayama, N., Nakatsukasa, M., Hayashi, T., \& Agata, K. (2005). Isolation of the choanocyte in the fresh water sponge, \textit{Ephydatia fluviatilis} and its lineage marker, \textit{Ef annexin}. \textit{Development, growth \& differentiation}, \textit{47}(4), 243-253. \ulhref{https://doi.org/10.1111/j.1440-169X.2005.00800.x}{10.1111/j.1440-169X.2005.00800.x}
		\end{itemize}
\end{enumerate}

\clearpage

\end{document}