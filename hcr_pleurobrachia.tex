\documentclass[10pt]{report}
\usepackage[utf8]{inputenc}

% DOCUMENT STYLE
\usepackage[a4paper,
  left=1.5cm, right=1.5cm, top=2.7cm, bottom=2cm,
  headsep=\dimexpr2.5cm-55pt\relax,
  headheight=55pt]{geometry}
\linespread{1.5}                                    % regulate line spacing
\usepackage{setspace}
\usepackage[parfill]{parskip}                       % remove indentation of first line
\usepackage[T1]{fontenc}
\usepackage{xcolor}
\usepackage{enumitem}
\usepackage{float}

% FONTS
\usepackage{fontspec}
\setmainfont{RobotoSerif}[
  Path = ./font/ ,
  Extension = .ttf ,
  UprightFont = *-Regular ,
  ItalicFont = *-Italic ,
  BoldFont = *-Bold ,
  BoldItalicFont = *-BoldItalic
]

% package to generate random textcomp
\usepackage{blindtext}

% FORMAT HEADERS AND FOOTERS
\usepackage{fancyhdr}
\pagestyle{fancy} % page style to fancy to print header
\rhead{\footnotesize \emph{Last update}: \today}
\lhead{\footnotesize \em \nouppercase{\leftmark{} \textendash{} \rightmark}}

% CHEMICAL FORMULAE AND SPECIAL SYMBOLS
\usepackage[version=4]{mhchem}
\usepackage{textcomp}
\usepackage{siunitx}
\sisetup{
  detect-all = true,                                    % detect and follow surrounding font changes
  range-phrase= \textendash{}, range-units = single,               % set range format
  per-mode = symbol,                                    % set "/" as format for fraction units
  uncertainty-mode = separate,                          % seprate uncertainty (+-) from number
  separate-uncertainty-units = single,                  % put units only after uncertainty
  group-digits = integer,                               % apply digit grouping just for the integer part 
  group-separator = {,},                                % separator between digits
  group-minimum-digits = 4,                             % set the minimun number of digits to apply goruping to 4
  bracket-unit-denominator = false,                     % do not put brackets in denominator
  list-units = single,                                  % in list mode, put units only at the end
  exponent-mode = threshold,                            % automatically typeset with scientific notation (default threshold of -3:3)
  print-unity-mantissa = false                          % do not print mantissa if it is 1
  %round-mode = places,
  %round-precision = 3
  }
\DeclareSIUnit{\nothing}{\relax}                  % define unit to typset just prefixes
\DeclareSIUnit{\molar}{M}                         % define unit for molar cocnentration (M)

\newcommand{\per}{$\times$}
\newcommand{\NA}{\textendash{}}
\newcommand{\doublecurlyquotes}[1]{“#1”}
\newcommand{\singlecurlyquotes}[1]{‘#1’}
\newcommand{\curlyapostrophe}{’}

% percentages
\newcommand{\hundredpercent}{\qty{100}{\percent}}
\newcommand{\fivepercent}{\qty{5}{\percent}}
\newcommand{\onepercent}{\qty{1}{\percent}}
\newcommand{\zeroonepercent}{\qty{0.1}{\percent}}

% volumes
\newcommand{\twohunmicrol}{\qty{200}{\ul}}
\newcommand{\fivehunmicrol}{\qty{500}{\ul}}
\newcommand{\onemil}{\qty{1}{\ml}}

% temperatures
\newcommand{\fourdegree}{\qty{4}{\degreeCelsius}}
\newcommand{\thirtysevendegree}{\qty{37}{\degreeCelsius}}
\newcommand{\minustwenty}{\qty{-20}{\degreeCelsius}}

% time
\newcommand{\tenmin}{\qty{10}{\minute}}
\newcommand{\quarter}{\qty{15}{\minute}}
\newcommand{\halfhour}{\qty{30}{\minute}}

% reagents
\newcommand{\pbs}{1\per{} PBS}
\newcommand{\hs}{0.25\per{} HS}
\newcommand{\ssct}{5\per{} SSCT}

% others
\newcommand{\qtytimes}[3]{#1 \per{} \qty{#2}{#3}}
\newcommand{\noexponentnum}[1]{\num[exponent-mode = fixed, fixed-exponent = 0]{#1}}

% DATES
\usepackage[datesep=/, calc, en-GB]{datetime2}
\usepackage{fmtcount}

% define the style for CV, that is: Jan 1st, 2024
\DTMnewdatestyle{cvdateformat}{%
  \renewcommand{\DTMdisplaydate}[4]{%
  \DTMshortmonthname{##2} \protect\ordinalnum{##3}, \number##1 }%
  \renewcommand{\DTMDisplaydate}{\DTMdisplaydate}%
}

% GLOSSARIES, ACRONYMS, AND ABBREVIATIONS
\usepackage[acronym, nonumberlist, toc, nomain, nopostdot, shortcuts]{glossaries-extra}
%\loadglsentries{sections/abbreviations}	
% define command to print brackets when first usage is within parenthesis
\newcommand*{\pac}[2][]{\ifglsused{#2}{\acs[#1]{#2}}{%
 \glsunset{#2}%
 \acl[#1]{#2} [\acs[#1]{#2}]}}

% DEFINE TITLE STYLES
\makeatletter 
\renewcommand\maketitle{
{\begin{center}
{\Large \bfseries \@title }\\
{\small \emph{First created}: \DTMdisplaydate{2024}{12}{5}{-1}\hfill%
\setstretch{1.0} \small \hfill \emph{Last update}: \today}
%{\Large  \@author}\\[4ex] 
\end{center}}} % Note the extra }
\makeatother

\usepackage{titlesec}
%\titleformat{⟨command⟩}[⟨shape⟩]{⟨format⟩}{⟨label⟩}{⟨sep⟩}{⟨before-code⟩}[⟨after-code⟩]
\titleformat{\section}{\large\bfseries\uppercase}{\thesection}{}{\centering}
\titleformat{\subsection}{\large\bfseries}{\thesubsection}{}{}

% DEFINE BUTTON-STYLE BOXES
%\usepackage{tcolorbox}
%\tcbset{on line}

%\newcommand{\insertboxes}[3]{{\bfseries \tcbox{#1} \tcbox{#2} \tcbox{#3}}}

% DEFINE LEFT EMPHASIS-RULE
\usepackage{mdframed}
\newmdenv[
  topline=false,
  bottomline=false,
  rightline=false,
  linecolor=gray
]{leftrule}
\newcommand{\alert}[1]{{\itshape \small \setstretch{1.0} \begin{leftrule} \textcolor{gray}{#1} \end{leftrule}}}

% STEP SYNTAX
% wash steps: #1=action; #2=time; #3=quantity; #4=reagent; #5=temperature
\newcommand{\step}[5]{\MakeUppercase #1 samples for \underline{#2} with #3 of \textbf{#4} at #5.}
\newcommand{\note}[1]{{\itshape \small \textcolor{gray}{Note: #1}}}


% DEFINE TITLE STYLES
\makeatletter 
\renewcommand\maketitle{
{\begin{center}
{\Large \bfseries \@title }\\
{\small \emph{First created}: \DTMdisplaydate{2025}{03}{25}{-1}\hfill%
\setstretch{1.0} \small \hfill \emph{Last update}: \today}
%{\Large  \@author}\\[4ex] 
\end{center}}} % Note the extra }
\makeatother

%-------------------------------------------------------------------------

\begin{document}

%\sisetup{parse-numbers = false}
\DTMsetdatestyle{cvdateformat}

\title{\vspace{-3em} Multiplexed mRNA \textit{in-situ} Hybridization Chain Reaction (HCR) in \emph{Pleurobrachia pileus} adult tissues}
\maketitle
\thispagestyle{plain}
\markboth{HCR in \emph{Pleurobrachia} tissues}{}

\section*{\vspace{-1em} \textendash{} Protocol \textendash{}}
\markright{Protocol}

\subsection*{Day 1 \textendash{} Sample preparation}

\alert{In each step, samples can be rocked on a nutator or on an orbital (horizontal) shaker.}

\begin{enumerate}[series = steps]
	\item Dissect the desired adult tissue in 100\% methanol.
	\item Wash samples for \underline{\quarter} each with a graded series of \fivehunmicrol{} (or \onemil) of \textbf{methanol (\qtylist{60;30}{\percent}) in PBST} at RT.
	\item \step{wash}{\quarter}{\fivehunmicrol{} (or \onemil)}{PBST}{RT}
	\item \step{rock}{\qty{2}{\minute}}{\fivehunmicrol{} (or \onemil)}{\qty{2.6}{\ug\per\ml} proteinase K}{RT}
	\item \step{wash}{two times without incubation}{\fivehunmicrol{} (or \onemil)}{PBST}{RT}
	\item \step{post-fix}{\halfhour}{\fivehunmicrol{} (or \onemil)}{\qty{3.7}{\percent} PFA}{RT}
\end{enumerate}

\bigskip\alert{Pre-heat probe-hybridization buffer to \thirtysevendegree{} (you will need \qty{700}{\ul} per tube [\fivehunmicrol for the pre-hybridization, and \twohunmicrol for the probe solution]).}

\begin{enumerate}[resume = steps]
	\item \step{wash}{\qtytimes{2}{10}{\minute}}{\fivehunmicrol{} (or \onemil)}{PBST}{RT}
\end{enumerate}

\subsection*{Day 1 \textendash{} HCR detection stage}
\begin{enumerate}[series = steps]
	\item \step{pre-hybridize}{\halfhour}{\fivehunmicrol}{pre-warmed probe hybridization buffer}{\thirtysevendegree}
\end{enumerate}

\bigskip\alert{If reusing probes, pre-warm probe solution to \thirtysevendegree, then skip to Step 3.}

\alert{At this step, samples can be stored in probe hybridization buffer at \minustwenty{} for several months.}

\begin{enumerate}[resume = steps]
	\item Prepare probe solution by adding \qty{0.8}{\ul} (\qty{0.8}{\pmol}) of each \textbf{probe set \qty{1}{\micro\molar} stock solution} to \twohunmicrol{} of \textbf{probe hybridization buffer} (final concentration of \qty{4}{\nano\molar}) at \thirtysevendegree.\\
	      \note{if using \fivehunmicrol{} of probe solution, add \qty{2}{\ul} (\qty{2}{\pmol}) of each probe set.}
	\item \step{incubate}{\qty{> 12}{\hour} (overnight)}{\twohunmicrol}{pre-warmed probe solution}{\thirtysevendegree}
\end{enumerate}

\bigskip\alert{Pre-heat probe wash buffer at \thirtysevendegree{} (you will need \fivehunmicrol per tube per 4 washes).}


\subsection*{Day 2 \textendash{} HCR amplification stage}

\alert{Pre-equilibrate amplification buffer to room temperature (you will need \qty{700}{\ul} per tube [\fivehunmicrol{} for pre-amplification, and \twohunmicrol{} for the hairpin solution]).}
\begin{enumerate}[series = steps]
	\item \step{wash}{\qtytimes{4}{20}{\minute}}{\fivehunmicrol}{probe wash buffer}{\thirtysevendegree}
\end{enumerate}

\bigskip\alert{Probe solution can be saved and reused for 2--3 times; store at \minustwenty.}

\begin{enumerate}[resume = steps]
	\item \step{wash}{\qtytimes{3}{5}{\minute}}{\fivehunmicrol{} (or \onemil)}{\ssct}{RT}
	\item \step{pre-amplify}{\halfhour}{\fivehunmicrol}{amplification buffer}{RT}
	\item Snap cool in separate tubes \qty{4}{\ul} of \textbf{hairpin H1} (\qty{30}{\pmol}) and \qty{4}{\ul} of \textbf{hairpin H2} (\qty{30}{\pmol}) \textbf{\qty{3}{\micro\molar} stock solutions}: heat at \qty{95}{\degreeCelsius} for \qty{90}{\s} and cool to RT in the dark for \halfhour.
	      %\note{doubling hairpin concentration can help to boost signal.}
\end{enumerate}

\bigskip\alert{If reusing hairpin solutions, heat at \qty{95}{\degreeCelsius} and cool at RT, then skip at Step 6.}

\begin{enumerate}[resume = steps]
	\item Prepare hairpin solution by adding \textbf{snap-cooled H1 hairpins} and \textbf{snap-cooled H2 hairpins} to \twohunmicrol{} of \textbf{amplification buffer} (final concentration of each hairpin of \qty{60}{\nano\molar}) at RT.\\
	      \note{if using \qty{100}{\ul} of amplification buffer, add \qty{2}{\ul} (\qty{15}{\pmol}) of each hairpin.}\\
	      \note{if using \qty{500}{\ul} of amplification buffer, add \qty{10}{\ul} (\qty{75}{\pmol}) of each hairpin.}
	\item \step{incubate}{\qty{>12}{\hour} (overnight)}{\twohunmicrol}{hairpin solution}{RT}
\end{enumerate}

\subsection*{Day 3 \textendash{} HCR conclusion}

\begin{enumerate}[series = steps]
	\item \step{wash}{\qtytimes{2}{5}{\minute}, \qtytimes{2}{30}{\minute}, and \qtytimes{1}{5}{\minute}}{\fivehunmicrol{} (or \onemil)}{\ssct}{RT}
\end{enumerate}

\bigskip\alert{dsDNA can be stained on the second \qty{30}{\minute} wash by using a 1:500 DAPI solution.}
\alert{Hairpin solution can be saved and reused for 2--3 times; store at \minustwenty.}
\alert{At this step, samples can be stored in the dark in \ssct for several days at room temperature.}

\subsection*{Day 3 \textendash{} Sample mounting}

\begin{enumerate}[series = steps]
	\item \step{incubate}{\qtyrange{30}{60}{\minute}}{\onemil}{\qty{50}{\percent} glycerol in \pbs}{RT}
	\item \step{incubate}{\qtyrange{30}{60}{\minute}}{\onemil}{\qty{75}{\percent} glycerol in \pbs}{RT}
\end{enumerate}

\bigskip\alert{Maintaining samples at pH 7.40 is critical for the stability of fluorophores and long-term storage of mRNA \textit{in-situ} HCR samples.}
\alert{At this step, samples can be stored in the dark in \qty{75}{\percent} glycerol for several month at \fourdegree.}

\begin{enumerate}[resume = steps]
	\item Collect samples from the \qty{75}{\percent} glycerol solution and place them on a cleaned (bridged, if necessary) slide.
	\item Add \qtyrange{\sim 15}{20}{\ul} of mounting medium (or VECTASHIELD\textregistered{} PLUS Antifade Mounting Medium with DAPI [H-2000]) directly on samples.
\end{enumerate}

\bigskip\alert{Adjust the amount of mounting medium according to the cover glass dimensions, the quantity of samples, and the thickness of the bridge.}

\begin{enumerate}[resume = steps]
	\item	Seal the slide with nail polish and store in the dark at \fourdegree.
\end{enumerate}

\clearpage

\section*{\textendash{} Recipes \textendash{}}
\setstretch{0.95}
\markright{Recipes}

\subsection*{\pbs{} with \qty{0.1}{\percent} Tween 20 (PBST or PTw)}

\begin{table}[H]
	\centering
	\begin{tabular}{r
		S[table-format = 4.1, exponent-mode = fixed, fixed-exponent = 0, round-mode = places, round-precision = 1]@{\,} % disable scientific notation
		l
		S[table-format = 3.2, table-align-text-post = false, exponent-mode = fixed, fixed-exponent = 0, round-mode = places, round-precision = 2] % disable scientific notation
		c
		}
		\textbf{Compound} & \multicolumn{2}{c}{\textbf{Quantity}} & \multicolumn{1}{c}{\textbf{\begin{tabular}[c]{@{}c@{}}Molar mass\\ (\unit{\g\per\mole})\end{tabular}}} & \multicolumn{1}{l}{\textbf{\begin{tabular}[c]{@{}c@{}}Final concentration\\ (for \qty{50}{\ml})\end{tabular}}}                        \\*[0.4cm]
		Tween 20          & 50                                    & \unit{\ul}                                                                                             & 1227.54                                                                                                        & \qty{0.1}{\percent} \\
		20\per{} PBS      & 5                                     & \unit{\ml}                                                                                             & \NA                                                                                                            & 1\per{}              \\
		Ultrapure water   & \multicolumn{2}{c}{\NA}               & \NA                                                                                                    & \NA
	\end{tabular}
\end{table}

\begin{enumerate}
	\item Add \qty{5}{\ml} of 20\per{} PBS to a graduated cylinder.
	\item Add \qty{50}{\ul} of Tween 20.
	\item Fill up to \qty{50}{\ml} with ultrapure water.
\end{enumerate}

\subsection*{5\per{} SSC \qty{0.1}{\percent} Tween 20 (SSCT)}

\begin{table}[H]
	\centering
	\begin{tabular}{r
		S[table-format = 4.1, exponent-mode = fixed, fixed-exponent = 0, round-mode = places, round-precision = 1]@{\,} % disable scientific notation
		l
		S[table-format = 3.2, table-align-text-post = false, exponent-mode = fixed, fixed-exponent = 0, round-mode = places, round-precision = 2] % disable scientific notation
		c
		}
		\textbf{Compound} & \multicolumn{2}{c}{\textbf{Quantity}} & \multicolumn{1}{c}{\textbf{\begin{tabular}[c]{@{}c@{}}Molar mass\\ (\unit{\g\per\mole})\end{tabular}}} & \multicolumn{1}{l}{\textbf{\begin{tabular}[c]{@{}c@{}}Final concentration\\ (for \qty{50}{\ml})\end{tabular}}}                        \\*[0.4cm]
		Tween 20          & 50                                    & \unit{\ul}                                                                                             & 1227.54                                                                                                        & \qty{0.1}{\percent} \\
		20\per{} SSC      & 12.5                                  & \unit{\ml}                                                                                             & \NA                                                                                                            & 5\per{}              \\
		Ultrapure water   & \multicolumn{2}{c}{\NA}               & \NA                                                                                                    & \NA
	\end{tabular}
\end{table}

\begin{enumerate}
	\item Add \qty{12.5}{\ml} of 20\per{} SSC to a graduated cylinder.
	\item Add \qty{50}{\ul} of Tween 20.
	\item Fill up to \qty{50}{\ml} with ultrapure water.
\end{enumerate}

\subsection*{Proteinase K working solution}

\begin{table}[H]
	\centering
	\begin{tabular}{r
		S[table-format = 4.2, exponent-mode = fixed, fixed-exponent = 0, round-mode = places, round-precision = 2]@{\,} % disable scientific notation
		l
		S[table-format = 3.2, table-align-text-post = false, exponent-mode = fixed, fixed-exponent = 0, round-mode = places, round-precision = 2] % disable scientific notation
		c
		}
		\textbf{Compound}                                                                                              & \multicolumn{2}{c}{\textbf{Quantity}} & \multicolumn{1}{c}{\textbf{\begin{tabular}[c]{@{}c@{}}Molar mass\\ (\unit{\g\per\mole})\end{tabular}}} & \multicolumn{1}{l}{\textbf{\begin{tabular}[c]{@{}c@{}}Final concentration\\ (for \qty{10}{\ml})\end{tabular}}}                         \\*[0.4cm]
		\multicolumn{1}{c}{\begin{tabular}[c]{@{}r@{}}\qty{20}{\mg\per\ml} proteinase K\\ stock solution\end{tabular}} & 1.3                                   & \unit{\ul}                                                                                             & \NA                                                                                                            & \qty{2.6}{\ug\per\ml} \\*[0.4cm]
		PBST                                                                                                           & \multicolumn{2}{c}{\NA}               & \NA                                                                                                    & \NA
	\end{tabular}
\end{table}

\begin{enumerate}
	\item Add \qty{1.3}{\ul} of \qty{20}{\mg\per\ml} proteinase K stock solution to \qty{8}{\ml} of PBST.
	\item Fill up to \qty{10}{\ml} with PBST.
\end{enumerate}

\subsection*{Urea-based probe hybridization buffer}

\begin{table}[H]
	\centering
	\begin{tabular}{r
		S[table-format = 4.2, exponent-mode = fixed, fixed-exponent = 0, round-mode = places, round-precision = 2]@{\,} % disable scientific notation
		l
		S[table-format = 3.2, table-align-text-post = false, exponent-mode = fixed, fixed-exponent = 0, round-mode = places, round-precision = 2] % disable scientific notation
		c
		}
		\textbf{Compound}                                                                                              & \multicolumn{2}{c}{\textbf{Quantity}} 						& \multicolumn{1}{c}{\textbf{\begin{tabular}[c]{@{}c@{}}Molar mass\\ (\unit{\g\per\mole})\end{tabular}}} & \multicolumn{1}{l}{\textbf{\begin{tabular}[c]{@{}c@{}}Final concentration\\ (for \qty{10}{\ml})\end{tabular}}}                         \\*[0.4cm]
		Urea																										   & 2.4                                   & \unit{\g}			& 60.06                                                                                            		 & \qty{4}{\molar} 																														  \\
		20\per{} SSC                                                                                                   & 2.5                                   & \unit{\ml}         & \NA{}                                                                                  				 & 5\per{}																																  \\
		\qty{1}{\molar} citric acid																			           & 90									   & \unit{\ul}			& 192.124																								 & \qty{9}{\milli\molar}																												  \\
		Tween-20																									   & 10									   & \unit{\ul}			& 1227.54																							     & \qty{0.1}{\percent}                                                                                                                    \\
		50\per{} Denhardts's solution																				   & 200								   & \unit{\ul}			& \NA{}																									 & 1\per{}																																  \\
		\qty{50}{\percent} dextran sulfate																	           & 2   								   & \unit{\ml}			& \NA{}																									 & \qty{10}{\percent}																													  \\
		\qty{10}{\mg\per\ml} heparin																				   & 50									   & \unit{\ul}			& \NA{}																									 & \qty{50}{\ug\per\ml}																													  \\
	\end{tabular}
\end{table}

\begin{enumerate}
	\item Add \qty{2.4}{\g} of urea to a becker.
	\item Combine all the other components fresh.
	\item Stir thoroughly to allow complete solubilization of urea.
	\item Fill up to \qty{10}{\ml} with ultrapure water.
\end{enumerate}

\subsection*{Urea-based probe wash buffer}

\begin{table}[H]
	\centering
	\begin{tabular}{r
		S[table-format = 4.2, exponent-mode = fixed, fixed-exponent = 0, round-mode = places, round-precision = 2]@{\,} % disable scientific notation
		l
		S[table-format = 3.2, table-align-text-post = false, exponent-mode = fixed, fixed-exponent = 0, round-mode = places, round-precision = 2] % disable scientific notation
		c
		}
		\textbf{Compound}                                                                                              & \multicolumn{2}{c}{\textbf{Quantity}} 						& \multicolumn{1}{c}{\textbf{\begin{tabular}[c]{@{}c@{}}Molar mass\\ (\unit{\g\per\mole})\end{tabular}}} & \multicolumn{1}{l}{\textbf{\begin{tabular}[c]{@{}c@{}}Final concentration\\ (for \qty{10}{\ml})\end{tabular}}}                         \\*[0.4cm]
		Urea																										   & 2.4                                   & \unit{\g}			& 60.06                                                                                            		 & \qty{4}{\molar} 																														  \\
		20\per{} SSC                                                                                                   & 2.5                                   & \unit{\ml}         & \NA{}                                                                                  				 & 5\per{}																																  \\
		\qty{1}{\molar} citric acid																			           & 90									   & \unit{\ul}			& 192.124																								 & \qty{9}{\milli\molar}																												  \\
		Tween-20																									   & 10									   & \unit{\ul}			& 1227.54																							     & \qty{0.1}{\percent}                                                                                                                    \\
		\qty{10}{\mg\per\ml} heparin																				   & 50									   & \unit{\ul}			& \NA{}																									 & \qty{50}{\ug\per\ml}																													  \\
	\end{tabular}
\end{table}

\begin{enumerate}
	\item Add \qty{2.4}{\g} of urea to a becker.
	\item Combine all the other components fresh.
	\item Stir thoroughly to allow complete solubilization of urea.
	\item Fill up to \qty{10}{\ml} with ultrapure water.
\end{enumerate}


\subsection*{1:500 DAPI staining solution}

\begin{table}[H]
	\centering
	\begin{tabular}{r
		S[table-format = 4.1, exponent-mode = fixed, fixed-exponent = 0, round-mode = places, round-precision = 1]@{\,} % disable scientific notation
		l
		S[table-format = 3.2, table-align-text-post = false, exponent-mode = fixed, fixed-exponent = 0, round-mode = places, round-precision = 2] % disable scientific notation
		c
		}
		\textbf{Compound} & \multicolumn{2}{c}{\textbf{Quantity}} & \multicolumn{1}{c}{\textbf{\begin{tabular}[c]{@{}c@{}}Molar mass\\ (\unit{\g\per\mole})\end{tabular}}} & \multicolumn{1}{l}{\textbf{\begin{tabular}[c]{@{}c@{}}Final concentration\\ (for \qty{500}{\ul})\end{tabular}}}         \\*[0.4cm]
		DAPI              & 1                                     & \unit{\ul}                                                                                             & \NA                                                                                                             & 1:500 \\
		\pbs{}            & 499                                   & \unit{\ul}                                                                                             & \NA                                                                                                             & \NA
	\end{tabular}
\end{table}

\clearpage

\section*{\textendash{} Resources \textendash{}}
\markright{Resources}

\begin{enumerate}
	\item \textbf{ISH/IHC in \textit{Mnemiopsis leidyi}.}
		\begin{itemize}
			\item Burkhardt, P., Colgren, J., Medhus, A., Digel, L., Naumann, B., Soto-Angel, J. J., ... \& Kittelmann, M. (2023). Syncytial nerve net in a ctenophore adds insights on the evolution of nervous systems. \textit{Science}, \textit{380}(6642), 293-297. \ulhref{https://doi.org/10.1126/science.ade5645}{10.1126/science.ade5645}
			\item Mitchell, D. G., Edgar, A., Mateu, J. R., Ryan, J. F., \& Martindale, M. Q. (2024). The ctenophore \textit{Mnemiopsis leidyi} deploys a rapid injury response dating back to the last common animal ancestor. \textit{Communications Biology}, \textit{7}(1), 203. \ulhref{https://doi.org/10.1038/s42003-024-05901-7}{10.1038/s42003-024-05901-7}
			\item Moroz, L. L., \& Kohn, A. B. (2015). Analysis of gene expression in neurons and synapses by multi-color \textit{in situ} hybridization. \textit{In Situ Hybridization Methods}, 293-317.
			\item Pang, K., \& Martindale, M. Q. (2008). Ctenophore whole-mount \textit{in situ} hybridization. \textit{Cold Spring Harbor Protocols}, \textit{2008}(11), pdb-prot5087. \ulhref{https://doi.org/10.1101/pdb.prot5087}{10.1101/pdb.prot5087}
			\item Sachkova, M. Y., Nordmann, E. L., Soto-Àngel, J. J., Meeda, Y., Górski, B., Naumann, B., ... \& Burkhardt, P. (2021). Neuropeptide repertoire and 3D anatomy of the ctenophore nervous system. \textit{Current Biology}, \textit{31}(23), 5274-5285. \ulhref{https://doi.org/10.1016/j.cub.2021.09.005}{10.1016/j.cub.2021.09.005}
		\end{itemize}
	\item \textbf{ISH with urea-based buffers}
		\begin{itemize}
			\item Aguillon, R., Rinsky, M., Simon-Blecher, N., Doniger, T., Appelbaum, L., \& Levy, O. (2024). CLOCK evolved in cnidaria to synchronize internal rhythms with diel environmental cues. \textit{Elife}, \textit{12}, RP89499. \ulhref{https://doi.org/10.7554/eLife.89499.4}{10.7554/eLife.89499.4}
			\item Sinigaglia, C., Thiel, D., Hejnol, A., Houliston, E., \& Leclère, L. (2018). A safer, urea-based in situ hybridization method improves detection of gene expression in diverse animal species. \textit{Developmental biology}, \textit{434}(1), 15-23. \ulhref{https://doi.org/10.1016/j.ydbio.2017.11.015}{10.1016/j.ydbio.2017.11.015}
		\end{itemize}
\end{enumerate}


\end{document}


